% !TeX encoding = UTF-8
% !TeX spellcheck = pt_BR
% !TeX root = 20232-CODDISC-JOAO-FERREIRA.tex
% ------------------------------------------------------------------------------

% ------------------------------------------------------------------------------
% Template para aulas da UNINASSAU
% Autor: João Ferreira da Silva Júnior
% Versão: 0.4.4 - 2023-07-19
% ------------------------------------------------------------------------------



% ------------------------------------------------------------------------------
% Procurar texto e alterar texto em arquivos, buscar em todo subdiretório
% find . -type f -name "*.tex" -exec sed -i 's/TEXTO ORIGINAL/TEXTO NOVO/g' {} +
% find . -type f -name "*.tex" -exec sed -i 's/Versão: 0.1 - 2020-03-27/Versão: 0.4.4 - 2023-07-19/g' {} +
%
% Procurar padrão no nome e alterar arquivos, buscar em todo subdiretório
% find . -name '*.*' -print0 | xargs -0 -n1 bash -c 'mv "$0" "${0/PADRÃO ANTIGO/SUBSTITUTO}"'
% ------------------------------------------------------------------------------



% ------------------------------------------------------------------------------
\documentclass[
  % ----------------------------------------------------------------------------
  11pt,
  % ----------------------------------------------------------------------------
  %aspectratio=43,
  aspectratio=169,
  % ----------------------------------------------------------------------------
  % Opções do pacote babel
  english,                    % idioma adicional para hifenização
  french,                     % idioma adicional para hifenização
  spanish,                    % idioma adicional para hifenização
  brazil                      % o último idioma é o principal do documento
  ]{beamer}
% ------------------------------------------------------------------------------





% ------------------------------------------------------------------------------
% ------------------------------------------------------------------------------
% ------------------------------------------------------------------------------
% ------------------------------------------------------------------------------
% ------------------------------------------------------------------------------
% Configurações visuais e de tema

% Extend beamer and a0poster for custom sized posters
%\usepackage[%
%  orientation=landscape,
%  size=custom,
%  width=16,
%  height=9,
%  scale=0.45
%  ]{beamerposter}

% Configura o avanço do slide para passos a cada objeto ou parágrafo do frame
\usepackage{beamerthemesplit}

% This command allows you to specify in a quite general way how a covered item should be rendered
\setbeamercovered{transparent}

% Navigation symbols are small icons that are shown on every slide by default
\setbeamertemplate{navigation symbols}{}

% Configura o tema e a cor
\usetheme{Madrid} % Madrid
\usecolortheme{seahorse} % seahorse
\usefonttheme{default}
% ------------------------------------------------------------------------------
% Definição de cores
\definecolor{black}{rgb}{0.0, 0.0, 0.0}
\definecolor{white}{rgb}{255,255,255}
\definecolor{red}{rgb}{255,0,0}
\definecolor{green}{rgb}{0,255,0}
\definecolor{blue}{rgb}{0,0,255}
% ------------------------------------------------------------------------------
% Define a aparência dos blocks
\setbeamertemplate{blocks}[default]
% ------------------------------------------------------------------------------



% ------------------------------------------------------------------------------
% ------------------------------------------------------------------------------
% ------------------------------------------------------------------------------
% ------------------------------------------------------------------------------
% ------------------------------------------------------------------------------
% Pacotes principais

% accept different input encodings (conversão automática dos acentos)
\usepackage[utf8]{inputenc}

% standard package for selecting font encodings
\usepackage[T1]{fontenc}
% ------------------------------------------------------------------------------



% ------------------------------------------------------------------------------
% ------------------------------------------------------------------------------
% ------------------------------------------------------------------------------
% ------------------------------------------------------------------------------
% ------------------------------------------------------------------------------
% Pacotes extras

% the package provides several macros to adjust boxed content
\usepackage{adjustbox}

% Algorithm2e is an environment for writing algorithms. An algorithm becomes a floating object (like figure, table, etc.).
\usepackage{algorithm2e}

% an extended set of fonts for use in mathematics
\usepackage{amsfonts}

% extension package for LaTeX that provides various features to facilitate writing math formulas and to improve the typographical quality of their output
\usepackage{amsmath}

% math symbols defined by LaTeX package
\usepackage{amssymb}
\usepackage{fontawesome5}

% Create PDF and SVG animations from graphics files and inline graphics
% \usepackage{animate}

% draw dash-lines in array/tabular environments
% \usepackage{arydshln}

% manages culturally-determined typographical (and other) rules, and hyphenation patterns for a wide range of languages
\usepackage[english,portuguese]{babel}
  \newcommand{\ingles}[1]{\foreignlanguage{english}{\emph{#1}}}
  \newcommand{\frances}[1]{\foreignlanguage{french}{\emph{#1}}}
  % \newcommand{\grego}[1]{\foreignlanguage{greek}{\emph{#1}}}
  \newcommand{\grego}[1]{\emph{#1}}
  \newcommand{\espanhol}[1]{\foreignlanguage{spanish}{\emph{#1}}}
  \newcommand{\portugues}[1]{\foreignlanguage{brazil}{\emph{#1}}}

% a symbol (dingbat) font and LaTeX macros for its use
\usepackage{bbding}

% enhances the quality of tables in LATEX, providing extra commands as well as behind-the-scenes optimisation
\usepackage{booktabs}

% adding colors to your text is supported by the color package
\usepackage{color}

% gives you Courier scaled at its normal size as the default Typewriter Font
% \usepackage{courier}

% this package provides user control over the layout of the three basic list environments: enumerate, itemize and description
% http://linorg.usp.br/CTAN/macros/latex/contrib/enumitem/enumitem.pdf
% \usepackage{enumitem}

% enhanced support for graphics
\usepackage{graphicx}

% is used to handle cross-referencing commands in LaTeX to produce hypertext links in the document
\usepackage{hyperref}

% this package gives you easy access to the Lorem Ipsum dummy text
\usepackage{lipsum}

% enables the user to typeset programs (programming code) within LATEX
\usepackage{listings}
  \lstset{postbreak=\raisebox{0ex}[0ex][0ex]{\ensuremath{\color{gray}\hookrightarrow\space}}}
  % Aqui tem um exemplo: https://github.com/abntex/abntex2/wiki/HowToFormatarCodigoFonte
  % https://tex.stackexchange.com/questions/64839/how-to-change-listing-caption

  % Listing -> Algorithm
  \renewcommand{\lstlistingname}{Algoritmo}

  % List of Listings -> List of Algorithms
  \renewcommand{\lstlistlistingname}{Lista de algoritmos}

% the Latin Modern family of fonts
\usepackage{lmodern}

% Multimedia inclusion package with Adobe Reader-9/X compatibility
% http://repositorios.cpai.unb.br/ctan/macros/latex/contrib/media9/doc/media9.pdf
\usepackage{media9}

% subliminal refinements towards typographical perfection
\usepackage{microtype}

% Multimedia inclusion package
% Substituído pelo media9
%\usepackage{movie15}
% \newcommand{\pdfmovie}[4]{\href{run:#1}{\framebox{\parbox[c][#3][c]{#2}{\center #4}}}}

% multicol defines a multicols environment which typesets text in multiple columns (up to a maximum of 10)
\usepackage{multicol}

% create tabular cells spanning multiple rows
\usepackage{multirow}

% provides some new list environments. Itemized and enumerated lists can be typesetted within paragraphs
\usepackage{paralist}

% draws high--quality function plots in normal or logarithmic scaling with a user-friendly interface directly in TeX
\usepackage{pgfplots}

% This package reads tab-separated numerical tables from input and generates code for pretty-printed LATEX-tabulars
\usepackage{pgfplotstable}
  \pgfplotsset{compat=1.14}

% provides the means to draw pie (and variant) charts, using PGF/TikZ
\usepackage{pgf-pie}

% this package simplifies the inclusion of external multi-page PDF documents in LaTeX documents
\usepackage{pdfpages}

% defines new commands \Centering, \RaggedLeft, and \RaggedRight and new environments Center, FlushLeft, and FlushRight, which set ragged text and are easily configurable to allow hyphenation
\usepackage{ragged2e}

% makes the page margins visible and hence
% \usepackage{showframe}

% A comprehensive (SI) units package
\usepackage{siunitx}

% provides hyphenatable spacing out (letterspacing), underlining, striking out, etc., using the TeX hyphenation algorithm to find the proper hyphens automatically
  \usepackage{soul}
  \makeatletter
  \let\HL\hl
  \renewcommand\hl{%
    \let\set@color\beamerorig@set@color
    \let\reset@color\beamerorig@reset@color
    \HL}
  \makeatother

% permit use of UTF-8 characters in soul
\usepackage{soulutf8}

% inclusão de Subfiguras https://www.ctan.org/pkg/subfig
\usepackage{subfig}

% for creating graphics programmatically
\usepackage{tikz}
  \usetikzlibrary{arrows,automata,shadows}

% select Adobe Times Roman (or equivalent) as default font
% \usepackage{times}

% Using a color stack for transparency with pdfTEX
\usepackage{transparent}

% provides many glyphs like male and female symbols and astronomical symbols, as well as the complete lasy font set and other odds and ends
\usepackage{wasysym}

% font handwrite
\usepackage{wedn}
  % inline font modify
  \newcommand{\setfont}[2]{{\fontfamily{#1}\selectfont #2}}
  % Usage
  %\setfont{wedn}{Texto...}

% adding colors to your text is supported by the color package
\usepackage{xcolor}
% ------------------------------------------------------------------------------



% ------------------------------------------------------------------------------
% ------------------------------------------------------------------------------
% ------------------------------------------------------------------------------
% ------------------------------------------------------------------------------
% ------------------------------------------------------------------------------
% Pacotes ABNTEX

% citações padrão ABNT
\usepackage[alf]{abntex2cite}
% ------------------------------------------------------------------------------



% ------------------------------------------------------------------------------
% ------------------------------------------------------------------------------
% ------------------------------------------------------------------------------
% ------------------------------------------------------------------------------
% ------------------------------------------------------------------------------
% Macros do documento
\def\universidade{UNINASSAU}

\def\idcurso{99}
\def\curso{Nome do Curso ou Tema}

\def\iddisciplina{99}
\def\disciplina{Nome da Disciplina ou Descrição Detalhada do Tópico em Estudo}

\def\professor{João Ferreira da Silva Júnior}
\def\professorfoto{img/sec01/professor.png}
\def\professoremail{joaoferreirape@gmail.com}
\def\professorsite{https://joaoferreirape.wordpress.com}
\def\professorlattes{http://lattes.cnpq.br/8904695743376784}

\def\semana{}
\def\diadaaula{}
\def\mesdaaula{}
\def\anodaaula{}
\def\semestre{2023-2}
% \def\data{\diadaaula~de~\mesdaaula~de~\anodaaula}
\def\data{Semestre: \semestre}

%\title{\iddisciplina~---~\disciplina}
\title[\disciplina]{\disciplina}

%\subtitle[\iddisciplina]{\textbf{\curso}}
%\subtitle[\iddisciplina]{\textbf{\disciplina}}
\subtitle[\curso]{\curso}

\subject{\semestre~---~\curso~---~\disciplina}

%\author[SILVA JUNIOR, J. F. $\therefore$]{\href{\professorlattes}{\textbf{\professor}}}
\author[SILVA JUNIOR, J. F. $\therefore$]{\href{\professorlattes}{\professor}}

\logo{\includegraphics[height=6ex]{img/logo/brasao.png}}

\institute[UNINASSAU]{\includegraphics[width=40ex]{img/logo/logo.png}}

\date{\data}
% ------------------------------------------------------------------------------



% ------------------------------------------------------------------------------
% ------------------------------------------------------------------------------
% ------------------------------------------------------------------------------
% ------------------------------------------------------------------------------
% ------------------------------------------------------------------------------
% configurações do pacote hyperref, precisa estar aqui devido as macros de dados do documento
\makeatletter
\hypersetup{% http://www.tug.org/applications/hyperref/manual.html#x1-120003.8
  anchorcolor={black},                      % set color of anchors
  % backref=false,                            % do bibliographical back references
  baseurl={\professorlattes},               % set base URL for document
  urlcolor={black},%{blue},                 % url color link
  % bookmarks={true},                         % make bookmarks
  bookmarksnumbered={false},                % put section numbers in bookmarks
  bookmarksopen={true},                     % open up bookmark tree
  bookmarksopenlevel={\maxdimen},           % level to which bookmarks are open
  % bookmarksdepth,                           % <no-value>, <number>, <part,chapter,section,...>
  bookmarkstype={toc},                      % to specify which ‘toc’ file to mimic
  % breaklinks={false},                       % allow links to break over lines
  % citebordercolor=0 1 0,                    % color of border around cites
  citecolor={black},%{green},               % color of citation links
  colorlinks={false},
  % false,                                    % color links
  % true,                                     % (tex4ht, dviwindo)
  % debug=false,                              % provide details of anchors defined; same as verbose
  % destlabel=false,                          % destinations are named by the first \label after the anchor creation
  draft={false},                            % do not do any hyperlinking
  % filebordercolor=0 .5 .5,                  % color of border around file links
  filecolor={black},%{cyan},                % color of file links
  final={true},                             % opposite of option draft
  % frenchlinks=false,                        % use small caps instead of color for links
  % hyperfigures=false,                       % make figures hyper links
  % hyperfootnotes=true,                      % set up hyperlinked footnotes
  % hyperindex=true,                          % set up hyperlinked indices
  % linkbordercolor=1 0 0,                    % color of border around links
  linkcolor={black},%{red},                 % color of links
  % linktocpage={false},                      % make page number, not text, be link on TOC, LOF and LOT
  % menubordercolor=1 0 0,                    % color of border around menu links
  menucolor={black},%{red},                 % color for menu links
  pageanchor={true},                        % put an anchor on every page
  % pagebackref={true},                       % backreference by page number
  pdfauthor={\professor~<\professoremail>}, % text for PDF Author field
  % pdfborder=0 0 1,                          % width of PDF link border
  % 0 0 0,                                    % (colorlinks)
  % pdfcenterwindow=true,                     % position the document window in the center of the screen
  pdfcreator={LaTeX with hyperref package, feet: \professor},
  % text for PDF Creator field
  pdfdisplaydoctitle=true,                  % display document title instead of file name in title bar
  pdfkeywords={{\universidade};{\curso};{\disciplina};{\professor}},
  % text for PDF Keywords field
  pdflang={pt-BR},                          % PDF language identifier (RFC 3066)
  pdfproducer={\professor~<\professoremail>},   % text for PDF Producer field
  % pdfstartpage=1,                           % page at which PDF document opens
  pdfsubject={\semestre~---~\curso~---~\disciplina},  % text for PDF Subject field
  pdftitle={\curso~---~\disciplina},    % text for PDF Title field
}
\makeatother
% ------------------------------------------------------------------------------



% ------------------------------------------------------------------------------
% ------------------------------------------------------------------------------
% ------------------------------------------------------------------------------
% ------------------------------------------------------------------------------
% ------------------------------------------------------------------------------
\begin{document}



% ------------------------------------------------------------------------------
% ------------------------------------------------------------------------------
% ------------------------------------------------------------------------------
% ------------------------------------------------------------------------------
% ------------------------------------------------------------------------------
% ------------------------------------------------------------------------------
% ------------------------------------------------------------------------------
% ------------------------------------------------------------------------------
% ------------------------------------------------------------------------------
% ------------------------------------------------------------------------------
\section*{Capa Externa}\label{sec:capa-externa}



{\usebackgroundtemplate{\includegraphics[width=\paperwidth,height=\paperheight,keepaspectratio]{img/logo/background-1.png}}
  \begin{frame}[plain]

    ~

  \end{frame}
} % usebackgroundtemplate



% ------------------------------------------------------------------------------
% ------------------------------------------------------------------------------
% ------------------------------------------------------------------------------
% ------------------------------------------------------------------------------
% ------------------------------------------------------------------------------
% ------------------------------------------------------------------------------
% ------------------------------------------------------------------------------
% ------------------------------------------------------------------------------
% ------------------------------------------------------------------------------
% ------------------------------------------------------------------------------
\section*{Capa}\label{sec:capa}



{\usebackgroundtemplate{{\transparent{0.1}\includegraphics[width=\paperwidth,height=\paperheight,keepaspectratio]{img/logo/background-1.png}}}
  \begin{frame}[plain,t]

    \titlepage

  \end{frame}
} % usebackgroundtemplate



% ------------------------------------------------------------------------------
% ------------------------------------------------------------------------------
% ------------------------------------------------------------------------------
% ------------------------------------------------------------------------------
% ------------------------------------------------------------------------------
% ------------------------------------------------------------------------------
% ------------------------------------------------------------------------------
% ------------------------------------------------------------------------------
% ------------------------------------------------------------------------------
% ------------------------------------------------------------------------------
\section*{Agenda}\label{sec:agenda}



\begin{frame}[allowframebreaks,t]\frametitle{Agenda ---}
%\begin{frame}[allowframebreaks,t]
%\begin{frame}[t]\frametitle{Agenda}

  \tableofcontents{}
  % \tableofcontents[sectionstyle=show/hide/shaded,subsectionstyle=show/hide/shaded]
  % \tableofcontents[sectionstyle=show,subsectionstyle=show]
  % \tableofcontents[hideallsubsections]
  % \tableofcontents[currentsection,currentsubsection]

\end{frame}



% ------------------------------------------------------------------------------
% ------------------------------------------------------------------------------
% ------------------------------------------------------------------------------
% ------------------------------------------------------------------------------
% ------------------------------------------------------------------------------
% ------------------------------------------------------------------------------
% ------------------------------------------------------------------------------
% ------------------------------------------------------------------------------
% ------------------------------------------------------------------------------
% ------------------------------------------------------------------------------
% !TeX encoding = UTF-8
% !TeX spellcheck = pt_BR
% !TeX root = 20232-CODDISC-JOAO-FERREIRA.tex
% ------------------------------------------------------------------------------

% ------------------------------------------------------------------------------
% Template para aulas da UNINASSAU
% Autor: João Ferreira da Silva Júnior
% Versão: 0.4.5 - 2023-09-17
% ------------------------------------------------------------------------------



% ------------------------------------------------------------------------------
% ------------------------------------------------------------------------------
% ------------------------------------------------------------------------------
% ------------------------------------------------------------------------------
% ------------------------------------------------------------------------------
% ------------------------------------------------------------------------------
% ------------------------------------------------------------------------------
% ------------------------------------------------------------------------------
% ------------------------------------------------------------------------------
% ------------------------------------------------------------------------------
\section*{Professor}\label{sec:professor}



% \begin{frame}[allowframebreaks,t]\frametitle{Professor ---}
% \begin{frame}[t]\frametitle{Professor}
\begin{frame}[t]

    \begin{columns}[onlytextwidth,T]

        \begin{column}[T]{0.48\linewidth}
            \begin{figure}[htb]
              \centering{}
              \includegraphics[width=0.5\linewidth,height=0.3\textheight,keepaspectratio]{\professorfoto}

              \textbf{\href{\professorlattes}{\professor}}
            \end{figure}

            \begin{itemize}
              \justifying{}
              \setlength\itemsep{1ex}
              \item \href{\professorsite}{\professorsite}
              \item \href{\professorlattes}{\professorlattes}
              \item Formado em Análise e Desenvolvimento de Sistemas em 2013 (UNOPAR).
              \item Pós-graduado em nível de especialista em Cloud Computing, 2021 (UPE).
              \item Pós-graduado em nível de especialista em I.A., 2022 (UPE).
            \end{itemize}
        \end{column}

        \begin{column}{0.48\linewidth}
            \small
            \begin{itemize}\small
                \justifying{}
                \setlength\itemsep{1ex}
                \item Mestre em Engenharia Elétrica, título obtido em 2017 (UFPE) com o tema Detecção de Perdas em Sistemas de Distribuição de Água.
                \item Doutorando em Engenharia Elétrica (UFPE).
                \item Doutorando em Engenharia da Computação (UPE).
                \item Professor da UNINASSAU nos cursos de Análise e Desenvolvimento de Sistemas, Ciência da Computação, Engenharia da Computação, Redes de Computadores, e Sistemas de Informação.
                \item Servidor Público na Compesa atuando como Analista de Tecnologia da Informação (Analista de Negócios, Administrador de Banco de Dados, Projetos de Inovação).
            \end{itemize}
        \end{column}

    \end{columns}

\end{frame}


% !TeX encoding = UTF-8
% !TeX spellcheck = pt_BR
% !TeX root = 20232-CODDISC-JOAO-FERREIRA.tex
% ------------------------------------------------------------------------------

% ------------------------------------------------------------------------------
% Template para aulas da UNINASSAU
% Autor: João Ferreira da Silva Júnior
% Versão: 0.4.5 - 2023-09-17
% ------------------------------------------------------------------------------



% ------------------------------------------------------------------------------
% ------------------------------------------------------------------------------
% ------------------------------------------------------------------------------
% ------------------------------------------------------------------------------
% ------------------------------------------------------------------------------
% ------------------------------------------------------------------------------
% ------------------------------------------------------------------------------
% ------------------------------------------------------------------------------
% ------------------------------------------------------------------------------
% ------------------------------------------------------------------------------
\section[Plano de Ensino]{Plano de Ensino}\label{sec:plano-ensino}



{\usebackgroundtemplate{{\transparent{0.1}\includegraphics[width=\paperwidth,height=\paperheight,keepaspectratio]{img/logo/background-1.png}}}
% \begin{frame}[allowframebreaks,t]%\frametitle{ ---}
  \begin{frame}[plain]%\frametitle{}

    \vfill

    \centering{}
    \Huge{\textbf{Plano de Ensino}}

    \vfill

\end{frame}
} % usebackgroundtemplate



% ------------------------------------------------------------------------------
% ------------------------------------------------------------------------------
% ------------------------------------------------------------------------------
% ------------------------------------------------------------------------------
% ------------------------------------------------------------------------------
\subsection[Ementa]{Ementa}\label{subsec:plano-ensino-ementa}



% \begin{frame}[allowframebreaks,t]\frametitle{Ementa ---}
\begin{frame}[t]\frametitle{Ementa}

  \justifying{}

  \textbf{Objetivo}

  Desenvolver no aluno as habilidades de analisar, avaliar, compreender, conceituar, conhecer, descrever, desenvolver, documentar, empregar, identificar e sintetizar assuntos importantes acerca \textbf{\curso}, através de estudo teórico e prático.

  \vspace*{\fill}

  \textbf{Competências}

  \begin{itemize}
    \justifying{}
    % \setlength\itemsep{1em}
    \item Competência 1.
    \item Competência 2.
    \item Competência 3.
    \item Competência 4.
    \item Competência 5.
  \end{itemize}

\end{frame}



% ------------------------------------------------------------------------------
% ------------------------------------------------------------------------------
% ------------------------------------------------------------------------------
% ------------------------------------------------------------------------------
% ------------------------------------------------------------------------------
\subsection[Conteúdo Programático]{Conteúdo Programático}\label{subsec:plano-ensino-conteudo}



% \begin{frame}[allowframebreaks,t]\frametitle{Conteúdo Programático ---}
\begin{frame}[t]\frametitle{Conteúdo Programático}

  \begin{columns}[onlytextwidth,T]

    \begin{column}{0.48\linewidth}
      \begin{itemize}
        \justifying{}
        \setlength\itemsep{1em}
        \item \textbf{Unidade I}:
        \begin{itemize}
          \justifying{}
          \setlength\itemsep{1em}
          \item Item.
          \begin{itemize}
            \justifying{}
            % \setlength\itemsep{1em}
            \item Subitem.
            \item Subitem.
          \end{itemize}
          \item Item.
          \begin{itemize}
            \justifying{}
            % \setlength\itemsep{1em}
            \item Subitem.
            \item Subitem.
          \end{itemize}
        \end{itemize}
      \end{itemize}
    \end{column}

    \begin{column}{0.48\linewidth}
      \begin{itemize}
        \justifying{}
        \setlength\itemsep{1em}
        \item \textbf{Unidade II}:
        \begin{itemize}
          \justifying{}
          \setlength\itemsep{1em}
          \item Item.
          \begin{itemize}
            \justifying{}
            % \setlength\itemsep{1em}
            \item Subitem.
            \item Subitem.
          \end{itemize}
          \item Item.
          \begin{itemize}
            \justifying{}
            % \setlength\itemsep{1em}
            \item Subitem.
            \item Subitem.
          \end{itemize}
        \end{itemize}
      \end{itemize}
    \end{column}

  \end{columns}

\end{frame}



% ------------------------------------------------------------------------------
% ------------------------------------------------------------------------------
% ------------------------------------------------------------------------------
% ------------------------------------------------------------------------------
% ------------------------------------------------------------------------------
\subsection[Metodologia de Ensino]{Metodologia de Ensino}\label{subsec:plano-ensino-metodologia}



% \begin{frame}[allowframebreaks,t]\frametitle{Metodologia de Ensino ---}
\begin{frame}[t]\frametitle{Metodologia de Ensino}

  \justifying{}
  A disciplina, dependendo de sua natureza, pode ser ministrada através de:

  \begin{columns}[onlytextwidth,T]

    \begin{column}{0.48\linewidth}
      \begin{itemize}
        \justifying{}
        \setlength\itemsep{1em}
        \item Conteúdos teóricos.
        \item Conteúdos práticos.
        \item Aulas de campo em instituições específicas.
        \item Exposições dialogadas.
      \end{itemize}
    \end{column}

    \begin{column}{0.48\linewidth}
      \begin{itemize}
        \justifying{}
        \setlength\itemsep{1em}
        \item Grupos de discussão.
        \item Seminários.
        \item Debates competitivos.
        \item Apresentação e discussão sobre:
        \begin{itemize}
          \justifying{}
          % \setlength\itemsep{1em}
          \item Recursos de multimídia e filmes.
          \item Casos práticos.
        \end{itemize}
      \end{itemize}
    \end{column}

  \end{columns}

  \vspace*{\fill}

  Assim, os conteúdos podem ser trabalhados mais dinamicamente, estimulando o senso crítico e científico dos alunos.

\end{frame}



% ------------------------------------------------------------------------------
% ------------------------------------------------------------------------------
% ------------------------------------------------------------------------------
% ------------------------------------------------------------------------------
% ------------------------------------------------------------------------------
\subsection[Metodologia de Avaliação]{Metodologia de Avaliação}\label{subsec:plano-ensino-avaliacao}



% \begin{frame}[allowframebreaks,t]\frametitle{Metodologia de Avaliação ---}
\begin{frame}[t]\frametitle{Metodologia de Avaliação}

  \begin{itemize}
    \justifying{}
    \setlength\itemsep{1em}
    \item No decorrer de cada período letivo são desenvolvidas 02 (duas) avaliações por disciplina, para efeito do cálculo da média parcial.
    \item A média parcial é calculada pela média aritmética das duas avaliações efetuadas. O aluno que alcançar a média parcial maior ou igual a 7,0 (sete) é considerado aprovado.
    \item O aluno que não alcançar a média parcial faz em exame final onde precisa alcançar média final maior ou igual a 5,0.
    \item São aplicadas avaliações dos tipos: provas teóricas, provas práticas, seminários, trabalhos individuais ou em grupo e outras atividades em classe e extraclasse.
    \item O exame final é, obrigatoriamente, prova escrita.
  \end{itemize}

\end{frame}



% ------------------------------------------------------------------------------
% ------------------------------------------------------------------------------
% ------------------------------------------------------------------------------
% ------------------------------------------------------------------------------
% ------------------------------------------------------------------------------
\subsection[Bibliografia Sugerida]{Bibliografia Sugerida}\label{subsec:plano-ensino-bibliografia}



% \begin{frame}[allowframebreaks,t]\frametitle{Bibliografia Básica ---}
\begin{frame}[t]\frametitle{Bibliografia Básica}

  \begin{itemize}
    \justifying{}
    \setlength\itemsep{1em}
    \item Exemplos de citações.
    \item Redes de Computadores. 5. ed. [S.l.]: Pearson Prentice Hall. \cite{Tanenbaum2011}
    \item Sistemas Operacionais Modernos. 4. ed. [S.l.]: Pearson Prentice Hall. \cite{Tanenbaum2016}
  \end{itemize}

\end{frame}



% \begin{frame}[allowframebreaks,t]\frametitle{Bibliografia Complementar ---}
\begin{frame}[t]\frametitle{Bibliografia Complementar}

  \begin{itemize}
    \justifying{}
    \setlength\itemsep{1em}
    \item Exemplos de citações.
    \item Redes de Computadores. 5. ed. [S.l.]: Pearson Prentice Hall. \cite{Tanenbaum2011}
    \item Sistemas Operacionais Modernos. 4. ed. [S.l.]: Pearson Prentice Hall. \cite{Tanenbaum2016}
  \end{itemize}

\end{frame}


\include{20232-CODDISC-SEC03-PLANEJAMENTO}

\include{20232-CODDISC-SEC04-INTRODUCAO}

\include{20232-CODDISC-SEC05-TEMA01}

\include{20232-CODDISC-SEC06-TEMA02}

\include{20232-CODDISC-SEC07-TEMA03}

\include{20232-CODDISC-SEC08-TEMA04}

\include{20232-CODDISC-SEC09-TEMA05}

\include{20232-CODDISC-SEC10-LISTAS}

\include{20232-CODDISC-SEC11-SEMINARIOS}

\include{20232-CODDISC-SEC12-PROJETOS}



% ------------------------------------------------------------------------------
% ------------------------------------------------------------------------------
% ------------------------------------------------------------------------------
% ------------------------------------------------------------------------------
% ------------------------------------------------------------------------------
% ------------------------------------------------------------------------------
% ------------------------------------------------------------------------------
% ------------------------------------------------------------------------------
% ------------------------------------------------------------------------------
% ------------------------------------------------------------------------------
\section[Agradecimento]{Agradecimento}\label{sec:agradecimentos}



{\usebackgroundtemplate{{\transparent{0.2}\includegraphics[width=\paperwidth,height=\paperheight,keepaspectratio]{img/logo/background-1.png}}}
  \begin{frame}[plain,t]

    \vfill
    \vspace*{0.15\paperheight}

    \begin{center}
      \Huge{\textbf{Obrigado!}}
    \end{center}

    \vspace*{0.15\paperheight}
    \vfill

    \begin{block}{\professor}
      \justifying{}
      \begin{itemize}
        \justifying{}
        \setlength\itemsep{1em}
        \item \href{mailto:\professoremail}{\professoremail}
        \item \href{\professorlattes}{\professorlattes}
        \item \href{professorsite}{\professorsite}
      \end{itemize}
    \end{block}

  \end{frame}
} % usebackgroundtemplate



% ------------------------------------------------------------------------------
% ------------------------------------------------------------------------------
% ------------------------------------------------------------------------------
% ------------------------------------------------------------------------------
% ------------------------------------------------------------------------------
% ------------------------------------------------------------------------------
% ------------------------------------------------------------------------------
% ------------------------------------------------------------------------------
% ------------------------------------------------------------------------------
% ------------------------------------------------------------------------------
\section[Referências]{Referências}\label{sec:referencias}



{\usebackgroundtemplate{{\transparent{0.1}\includegraphics[width=\paperwidth,height=\paperheight,keepaspectratio]{img/logo/background-1.png}}}
  \begin{frame}[plain,allowframebreaks,t]%\frametitle{Referências ---}
  % \begin{frame}[plain,t]\frametitle{Referências}

    \begin{center}
      \Huge{\textbf{Referências}}
    \end{center}

    % \bibliographystyle{apalike}
    \bibliography{library}

  \end{frame}
} % usebackgroundtemplate



% ------------------------------------------------------------------------------
% ------------------------------------------------------------------------------
% ------------------------------------------------------------------------------
% ------------------------------------------------------------------------------
% ------------------------------------------------------------------------------
% ------------------------------------------------------------------------------
% ------------------------------------------------------------------------------
% ------------------------------------------------------------------------------
% ------------------------------------------------------------------------------
% ------------------------------------------------------------------------------
\section*{Capa Final}\label{sec:capa-final}



{\usebackgroundtemplate{\includegraphics[width=\paperwidth,height=\paperheight,keepaspectratio]{img/logo/background-3.png}}
  \begin{frame}[plain]

    ~

  \end{frame}
} % usebackgroundtemplate



\end{document}
