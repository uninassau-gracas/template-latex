% !TeX encoding = UTF-8
% !TeX spellcheck = pt_BR
% !TeX root = 20232-CODDISC-JOAO-FERREIRA.tex
% ------------------------------------------------------------------------------

% ------------------------------------------------------------------------------
% Template para aulas da UNINASSAU
% Autor: João Ferreira da Silva Júnior
% Versão: 0.4.5 - 2023-09-17
% ------------------------------------------------------------------------------



% ------------------------------------------------------------------------------
% ------------------------------------------------------------------------------
% ------------------------------------------------------------------------------
% ------------------------------------------------------------------------------
% ------------------------------------------------------------------------------
% ------------------------------------------------------------------------------
% ------------------------------------------------------------------------------
% ------------------------------------------------------------------------------
% ------------------------------------------------------------------------------
% ------------------------------------------------------------------------------
\section[Plano de Ensino]{Plano de Ensino}\label{sec:plano-ensino}



{\usebackgroundtemplate{{\transparent{0.1}\includegraphics[width=\paperwidth,height=\paperheight,keepaspectratio]{img/logo/background-1.png}}}
% \begin{frame}[allowframebreaks,t]%\frametitle{ ---}
  \begin{frame}[plain]%\frametitle{}

    \vfill

    \centering{}
    \Huge{\textbf{Plano de Ensino}}

    \vfill

\end{frame}
} % usebackgroundtemplate



% ------------------------------------------------------------------------------
% ------------------------------------------------------------------------------
% ------------------------------------------------------------------------------
% ------------------------------------------------------------------------------
% ------------------------------------------------------------------------------
\subsection[Ementa]{Ementa}\label{subsec:plano-ensino-ementa}



% \begin{frame}[allowframebreaks,t]\frametitle{Ementa ---}
\begin{frame}[t]\frametitle{Ementa}

  \justifying{}

  \textbf{Objetivo}

  Desenvolver no aluno as habilidades de analisar, avaliar, compreender, conceituar, conhecer, descrever, desenvolver, documentar, empregar, identificar e sintetizar assuntos importantes acerca \textbf{\curso}, através de estudo teórico e prático.

  \vspace*{\fill}

  \textbf{Competências}

  \begin{itemize}
    \justifying{}
    % \setlength\itemsep{1em}
    \item Competência 1.
    \item Competência 2.
    \item Competência 3.
    \item Competência 4.
    \item Competência 5.
  \end{itemize}

\end{frame}



% ------------------------------------------------------------------------------
% ------------------------------------------------------------------------------
% ------------------------------------------------------------------------------
% ------------------------------------------------------------------------------
% ------------------------------------------------------------------------------
\subsection[Conteúdo Programático]{Conteúdo Programático}\label{subsec:plano-ensino-conteudo}



% \begin{frame}[allowframebreaks,t]\frametitle{Conteúdo Programático ---}
\begin{frame}[t]\frametitle{Conteúdo Programático}

  \begin{columns}[onlytextwidth,T]

    \begin{column}{0.48\linewidth}
      \begin{itemize}
        \justifying{}
        \setlength\itemsep{1em}
        \item \textbf{Unidade I}:
        \begin{itemize}
          \justifying{}
          \setlength\itemsep{1em}
          \item Item.
          \begin{itemize}
            \justifying{}
            % \setlength\itemsep{1em}
            \item Subitem.
            \item Subitem.
          \end{itemize}
          \item Item.
          \begin{itemize}
            \justifying{}
            % \setlength\itemsep{1em}
            \item Subitem.
            \item Subitem.
          \end{itemize}
        \end{itemize}
      \end{itemize}
    \end{column}

    \begin{column}{0.48\linewidth}
      \begin{itemize}
        \justifying{}
        \setlength\itemsep{1em}
        \item \textbf{Unidade II}:
        \begin{itemize}
          \justifying{}
          \setlength\itemsep{1em}
          \item Item.
          \begin{itemize}
            \justifying{}
            % \setlength\itemsep{1em}
            \item Subitem.
            \item Subitem.
          \end{itemize}
          \item Item.
          \begin{itemize}
            \justifying{}
            % \setlength\itemsep{1em}
            \item Subitem.
            \item Subitem.
          \end{itemize}
        \end{itemize}
      \end{itemize}
    \end{column}

  \end{columns}

\end{frame}



% ------------------------------------------------------------------------------
% ------------------------------------------------------------------------------
% ------------------------------------------------------------------------------
% ------------------------------------------------------------------------------
% ------------------------------------------------------------------------------
\subsection[Metodologia de Ensino]{Metodologia de Ensino}\label{subsec:plano-ensino-metodologia}



% \begin{frame}[allowframebreaks,t]\frametitle{Metodologia de Ensino ---}
\begin{frame}[t]\frametitle{Metodologia de Ensino}

  \justifying{}
  A disciplina, dependendo de sua natureza, pode ser ministrada através de:

  \begin{columns}[onlytextwidth,T]

    \begin{column}{0.48\linewidth}
      \begin{itemize}
        \justifying{}
        \setlength\itemsep{1em}
        \item Conteúdos teóricos.
        \item Conteúdos práticos.
        \item Aulas de campo em instituições específicas.
        \item Exposições dialogadas.
      \end{itemize}
    \end{column}

    \begin{column}{0.48\linewidth}
      \begin{itemize}
        \justifying{}
        \setlength\itemsep{1em}
        \item Grupos de discussão.
        \item Seminários.
        \item Debates competitivos.
        \item Apresentação e discussão sobre:
        \begin{itemize}
          \justifying{}
          % \setlength\itemsep{1em}
          \item Recursos de multimídia e filmes.
          \item Casos práticos.
        \end{itemize}
      \end{itemize}
    \end{column}

  \end{columns}

  \vspace*{\fill}

  Assim, os conteúdos podem ser trabalhados mais dinamicamente, estimulando o senso crítico e científico dos alunos.

\end{frame}



% ------------------------------------------------------------------------------
% ------------------------------------------------------------------------------
% ------------------------------------------------------------------------------
% ------------------------------------------------------------------------------
% ------------------------------------------------------------------------------
\subsection[Metodologia de Avaliação]{Metodologia de Avaliação}\label{subsec:plano-ensino-avaliacao}



% \begin{frame}[allowframebreaks,t]\frametitle{Metodologia de Avaliação ---}
\begin{frame}[t]\frametitle{Metodologia de Avaliação}

  \begin{itemize}
    \justifying{}
    \setlength\itemsep{1em}
    \item No decorrer de cada período letivo são desenvolvidas 02 (duas) avaliações por disciplina, para efeito do cálculo da média parcial.
    \item A média parcial é calculada pela média aritmética das duas avaliações efetuadas. O aluno que alcançar a média parcial maior ou igual a 7,0 (sete) é considerado aprovado.
    \item O aluno que não alcançar a média parcial faz em exame final onde precisa alcançar média final maior ou igual a 5,0.
    \item São aplicadas avaliações dos tipos: provas teóricas, provas práticas, seminários, trabalhos individuais ou em grupo e outras atividades em classe e extraclasse.
    \item O exame final é, obrigatoriamente, prova escrita.
  \end{itemize}

\end{frame}



% ------------------------------------------------------------------------------
% ------------------------------------------------------------------------------
% ------------------------------------------------------------------------------
% ------------------------------------------------------------------------------
% ------------------------------------------------------------------------------
\subsection[Bibliografia Sugerida]{Bibliografia Sugerida}\label{subsec:plano-ensino-bibliografia}



% \begin{frame}[allowframebreaks,t]\frametitle{Bibliografia Básica ---}
\begin{frame}[t]\frametitle{Bibliografia Básica}

  \begin{itemize}
    \justifying{}
    \setlength\itemsep{1em}
    \item Exemplos de citações.
    \item Redes de Computadores. 5. ed. [S.l.]: Pearson Prentice Hall. \cite{Tanenbaum2011}
    \item Sistemas Operacionais Modernos. 4. ed. [S.l.]: Pearson Prentice Hall. \cite{Tanenbaum2016}
  \end{itemize}

\end{frame}



% \begin{frame}[allowframebreaks,t]\frametitle{Bibliografia Complementar ---}
\begin{frame}[t]\frametitle{Bibliografia Complementar}

  \begin{itemize}
    \justifying{}
    \setlength\itemsep{1em}
    \item Exemplos de citações.
    \item Redes de Computadores. 5. ed. [S.l.]: Pearson Prentice Hall. \cite{Tanenbaum2011}
    \item Sistemas Operacionais Modernos. 4. ed. [S.l.]: Pearson Prentice Hall. \cite{Tanenbaum2016}
  \end{itemize}

\end{frame}
