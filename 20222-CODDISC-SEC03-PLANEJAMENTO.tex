% !TeX encoding = UTF-8
% !TeX spellcheck = pt_BR
% !TeX root = 20222-CODDISC-JOAO-FERREIRA.tex
% ------------------------------------------------------------------------------

% ------------------------------------------------------------------------------
% Template para aulas da UNINASSAU
% Autor: João Ferreira da Silva Júnior
% Versão: 0.4 - 2022-08-19
% ------------------------------------------------------------------------------



% ------------------------------------------------------------------------------
% ------------------------------------------------------------------------------
% ------------------------------------------------------------------------------
% ------------------------------------------------------------------------------
% ------------------------------------------------------------------------------
% ------------------------------------------------------------------------------
% ------------------------------------------------------------------------------
% ------------------------------------------------------------------------------
% ------------------------------------------------------------------------------
% ------------------------------------------------------------------------------
\section[Planejamento da Disciplina]{Planejamento da Disciplina}\label{sec:planejamento}



{\usebackgroundtemplate{{\transparent{0.1}\includegraphics[width=\paperwidth,height=\paperheight,keepaspectratio]{img/logo/background-branco.png}}}
  % \begin{frame}[allowframebreaks,t]%\frametitle{ ---}
  \begin{frame}[plain]%\frametitle{}

    \vfill

    \centering{}
    \Huge{\textbf{Planejamento da Disciplina}}

    \vfill

  \end{frame}
} % usebackgroundtemplate



% ------------------------------------------------------------------------------
% ------------------------------------------------------------------------------
% ------------------------------------------------------------------------------
% ------------------------------------------------------------------------------
% ------------------------------------------------------------------------------
\subsection[Calendário Acadêmico]{Calendário Acadêmico}\label{subsec:planejamento-calendario}



% \begin{frame}[allowframebreaks,t]\frametitle{Calendário Acadêmico ---}
\begin{frame}[t]\frametitle{Calendário Acadêmico}

  \begin{itemize}
    \justifying{}
    \setlength\itemsep{1em}
    \item 01/01/2001 a 01/01/2001 --- início das aulas.
    \item 02/02/2002 a 02/02/2002 --- lista de exercícios, seminários e projetos.
    \item 02/02/2002 a 02/02/2002 --- 1ª avaliação.
    \item 03/03/2003 a 03/03/2003 --- lista de exercícios, seminários e projetos.
    \item 03/03/2003 a 03/03/2003 --- 2ª avaliação.
    \item 04/04/2004 a 04/04/2004 --- 2ª chamada.
    \item 05/05/2005 a 05/05/2005 --- avaliação final.
    \item O detalhamento e datas específicas constam no portal do aluno.
  \end{itemize}

\end{frame}



% ------------------------------------------------------------------------------
% ------------------------------------------------------------------------------
% ------------------------------------------------------------------------------
% ------------------------------------------------------------------------------
% ------------------------------------------------------------------------------
\subsection[Participação]{Participação}\label{subsec:planejamento-participacao}



% \begin{frame}[allowframebreaks,t]\frametitle{Participação ---}
\begin{frame}[t]\frametitle{Participação}

  \begin{itemize}
    \justifying{}
    \setlength\itemsep{1em}
    \item Frequência mínima de 75\% das aulas, determinação institucional.
    \item Participação em aula e interação útil entre os alunos.
    \item Não às conversas que causam dispersão.
    \item Questionamentos pertinentes e ajuda aos demais colegas de turma.
    \item Adequação ao conteúdo ministrado e contexto da disciplina.
  \end{itemize}

\end{frame}



% ------------------------------------------------------------------------------
% ------------------------------------------------------------------------------
% ------------------------------------------------------------------------------
% ------------------------------------------------------------------------------
% ------------------------------------------------------------------------------
\subsection[Lista de Exercícioss]{Lista de Exercícioss}\label{subsec:planejamento-lista}



% \begin{frame}[allowframebreaks,t]\frametitle{Lista de Exercícioss ---}
\begin{frame}[t]\frametitle{Lista de Exercícioss}

  \begin{itemize}
    \justifying{}
    \setlength\itemsep{1em}
    \item Poderá, ou não, haver listas de exercícios, que irão valer ponto e servirão como base de estudo, mas não de todo o conteúdo ministrado, para as avaliações.
    \item A eventual pontuação por lista de exercício será avaliada e computada individualmente.
    \item A(s) lista(s) de exercícios aplicadas até a primeira avaliação, terá(ão) conteúdo relacionado e contará eventual pontuação para esta.
    \item A(s) lista(s) de exercícios apicadas após a primeira avaliação e até a segunda, terá(ão) conteúdo relacionado e contará eventual pontuação para esta.
    \item Nenhuma eventual pontuação valerá para a segunda chamada ou avaliação final.
    \item A(s) lista(s) de exercício valerá(ão) até 10\% (dez por cento) da respectiva avaliação.
  \end{itemize}

\end{frame}



% ------------------------------------------------------------------------------
% ------------------------------------------------------------------------------
% ------------------------------------------------------------------------------
% ------------------------------------------------------------------------------
% ------------------------------------------------------------------------------
\subsection[Seminários]{Seminários}\label{subsec:planejamento-seminarios}



% \begin{frame}[allowframebreaks,t]\frametitle{Seminários ---}
\begin{frame}[t]\frametitle{Seminários}
  
  \begin{itemize}
    \justifying{}
    \setlength\itemsep{1em}
    \item Poderá, ou não, haver apresentação de seminários, que irão valer ponto e servirão como base de estudo, mas não de todo conteúdo ministrado, para as avaliações.
    \item A eventual pontuação por seminário será avaliada e computada individualmente.
    \item Poderá ser feito individualmente ou em grupo de até 5 (cinco) membros, mas atentem de que aleatoriamente serão feitos questionamentos a qualquer um dos membros, e, na incapacidade de resposta, a falha será atribuída a todo o grupo.
    \item O seminário deverá ser apresentado para toda a turma quando serão observados e avaliados os aspectos do tempo de apresentação; conteúdo; clareza; domínio; preparo geral e recursos utilizados.
    \item A nota atribuída valerá até 10\% (dez por cento) da respectiva avaliação.
  \end{itemize}
  
\end{frame}



% ------------------------------------------------------------------------------
% ------------------------------------------------------------------------------
% ------------------------------------------------------------------------------
% ------------------------------------------------------------------------------
% ------------------------------------------------------------------------------
\subsection[Projeto]{Projeto}\label{subsec:planejamento-projeto}



% \begin{frame}[allowframebreaks,t]\frametitle{Projeto ---}
\begin{frame}[t]\frametitle{Projeto}
  
  \begin{itemize}
    \justifying{}
    \setlength\itemsep{1em}
    \item Poderá, ou não, haver projeto, voltado à aplicação dos conceitos aprendidos em sala de aula, nas listas de exercícios, nos seminários e na bibliografia que consta no Plano de Ensino.
    \item A eventual pontuação por projeto será avaliada e computada individualmente.
    \item Poderá ser feito individualmente ou em grupo de até 5 (cinco) membros, mas atentem de que aleatoriamente serão feitos questionamentos a qualquer um dos membros, e, na incapacidade de resposta, a falha será atribuída a todo o grupo.
    \item O projeto será avaliado em apresentação para toda a turma, quando deverão ser demonstrados todos os processos desenvolvidos na pesquisa e elaboração, bem como o eventual funcionamento do projeto.
    \item A nota atribuída valerá até 20\% (vinte por cento) da respectiva avaliação.
  \end{itemize}
  
\end{frame}



% ------------------------------------------------------------------------------
% ------------------------------------------------------------------------------
% ------------------------------------------------------------------------------
% ------------------------------------------------------------------------------
% ------------------------------------------------------------------------------
\subsection[Avaliações]{Avaliações}\label{subsec:planejamento-avaliacoes}



% \begin{frame}[allowframebreaks,t]\frametitle{Avaliações ---}
\begin{frame}[t]\frametitle{Avaliações}

  \begin{columns}[onlytextwidth,T]

    \begin{column}{0.48\linewidth}
      \begin{block}{1ª e 2ª avaliações:}
        \begin{itemize}
            \justifying{}
            %\setlength\itemsep{1em}
            \item Conteúdo estudado em cada Unidade:
            \begin{itemize}
              \justifying{}
              \item Bibliografia.
              \item Listas de exercícios.
              \item Seminários que foram apresentados.
              \item Projetos que foram propostos.
            \end{itemize}
            \item Cálculo da nota:
            \begin{itemize}
              \justifying{}
              \item Listas de exercícios até +10\%, se houver.
              \item Seminários até +10\%, se houver.
              \item Projetos até +20\%, se houver.
              \item Avaliação até 60\%, com todos os extras.
              \item Avaliação até 100\%, sem os extras.
            \end{itemize}
        \end{itemize}
      \end{block}
    \end{column}

    \begin{column}{0.48\linewidth}
        \begin{block}{2ª chamada e avaliação final:}
          \begin{itemize}
            \justifying{}
            %\setlength\itemsep{1em}
            \item Conteúdo estudado em ambas Unidades:
            \begin{itemize}
              \justifying{}
              \item Bibliografia.
              \item Listas de exercícios.
              \item Seminários que foram apresentados.
              \item Projetos que foram propostos.
            \end{itemize}
            \item Cálculo da nota:
            \begin{itemize}
              \justifying{}
              \item Não haverá atividade extra.
              \item Qualquer ponto de atividade extra, feita anteriormente, não será computado para as notas da 2ª chamada ou avaliação final.
              \item A nota de cada avaliação valerá 100\%.
            \end{itemize}
          \end{itemize}
        \end{block}
    \end{column}

  \end{columns}

\end{frame}



% ------------------------------------------------------------------------------
% ------------------------------------------------------------------------------
% ------------------------------------------------------------------------------
% ------------------------------------------------------------------------------
% ------------------------------------------------------------------------------
\subsection[Comunicação da Turma]{Comunicação da Turma}\label{subsec:planejamento-comunicacao-turma}



% \begin{frame}[allowframebreaks,t]\frametitle{Comunicação da Turma ---}
\begin{frame}[t]\frametitle{Comunicação da Turma}

  \begin{figure}[htb]
    \centering{}
    \includegraphics[width=0.07\linewidth]{img/sec03/telegram.png}

    \textbf{\href{https://telegram.org}{Telegram}}
  \end{figure}

  \begin{columns}[onlytextwidth,T]

    \begin{column}{0.48\linewidth}
      \begin{block}{Vantagens:}
        \begin{itemize}
          \justifying{}
          % \setlength\itemsep{1em}
          \item Grupo pode ter muitos membros e a troca de arquivos é bem otimizada.
          \item Histórico permanente, quem entra depois poderá ver todas as mensagens já trocadas no grupo.
          \item Ótimo aplicativo para dispositivos móveis, desktop e web.
        \end{itemize}
      \end{block}
    \end{column}

    \begin{column}{0.48\linewidth}
      \begin{block}{Disponível para:}
        \begin{itemize}
          \justifying{}
          % \setlength\itemsep{1em}
          \item \href{https://telegram.org/dl/android}{Android}
          \item \href{https://telegram.org/dl/ios}{iPhone/iPad}
          \item \href{https://desktop.telegram.org}{PC/Mac/Linux}
          \item \href{https://web.telegram.org}{Web}
          \item \href{https://macos.telegram.org}{macOS}
          \item \href{https://web.telegram.org}{https://web.telegram.org}
        \end{itemize}
      \end{block}
    \end{column}

  \end{columns}

\end{frame}
