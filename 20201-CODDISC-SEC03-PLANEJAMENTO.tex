% !TeX encoding = UTF-8
% !TeX spellcheck = pt_BR
% !TeX root = 20201-CODDISC-JOAO-FERREIRA.tex
% ------------------------------------------------------------------------------

% ------------------------------------------------------------------------------
% Template para aulas da UNINASSAU
% Autor: João Ferreira da Silva Júnior
% Versão: 0.1 - 2020-03-27
% ------------------------------------------------------------------------------



% ------------------------------------------------------------------------------
% ------------------------------------------------------------------------------
% ------------------------------------------------------------------------------
% ------------------------------------------------------------------------------
% ------------------------------------------------------------------------------
% ------------------------------------------------------------------------------
% ------------------------------------------------------------------------------
% ------------------------------------------------------------------------------
% ------------------------------------------------------------------------------
% ------------------------------------------------------------------------------
\section[Planejamento da Disciplina]{Planejamento da Disciplina}\label{sec:planejamento}



{\usebackgroundtemplate{{\transparent{0.2}\includegraphics[width=\paperwidth,height=\paperheight,keepaspectratio]{img/logo/background-branco.png}}}
  % \begin{frame}[allowframebreaks,t]%\frametitle{ ---}
  \begin{frame}[plain]%\frametitle{}

    \vfill
    \centering

    \addtobeamertemplate{block begin}{\pgfsetfillopacity{0.5}}{\pgfsetfillopacity{1}}
    % \setbeamercolor{block title}{use=structure,fg=white,bg=blue}
    \setbeamercolor{block body}{use=structure,fg=black,bg=blue!50!white}
    \begin{block}{}
      \centering{}
      \Huge{\textbf{Planejamento da Disciplina}}
    \end{block}

    \vfill

  \end{frame}
} % usebackgroundtemplate



% ------------------------------------------------------------------------------
% ------------------------------------------------------------------------------
% ------------------------------------------------------------------------------
% ------------------------------------------------------------------------------
% ------------------------------------------------------------------------------
\subsection[Calendário Acadêmico]{Calendário Acadêmico}\label{subsec:planejamento-calendario-academico}



% \begin{frame}[allowframebreaks,t]\frametitle{Calendário Acadêmico ---}
\begin{frame}[t]\frametitle{Calendário Acadêmico}

  \begin{itemize}
    \justifying{}
    \setlength\itemsep{1em}
    \item Objetivo 1.
    \item Objetivo 2.
    \item Objetivo 3.
  \end{itemize}

\end{frame}



% ------------------------------------------------------------------------------
% ------------------------------------------------------------------------------
% ------------------------------------------------------------------------------
% ------------------------------------------------------------------------------
% ------------------------------------------------------------------------------
\subsection[Participação]{Participação}\label{subsec:planejamento-participacao}



% \begin{frame}[allowframebreaks,t]\frametitle{Participação ---}
\begin{frame}[t]\frametitle{Participação}

  \begin{itemize}
    \justifying{}
    \setlength\itemsep{1em}
    \item Item 1.
    \item Item 2.
    \item Item 3.
  \end{itemize}

\end{frame}



% ------------------------------------------------------------------------------
% ------------------------------------------------------------------------------
% ------------------------------------------------------------------------------
% ------------------------------------------------------------------------------
% ------------------------------------------------------------------------------
\subsection[Lista de Exercícios]{Lista de Exercícios}\label{subsec:planejamento-lista-exercicios}



% \begin{frame}[allowframebreaks,t]\frametitle{Lista de Exercícios ---}
\begin{frame}[t]\frametitle{Lista de Exercícios}

  \begin{itemize}
    \justifying{}
    \setlength\itemsep{1em}
    \item Item 1.
    \item Item 2.
    \item Item 3.
  \end{itemize}

\end{frame}



% ------------------------------------------------------------------------------
% ------------------------------------------------------------------------------
% ------------------------------------------------------------------------------
% ------------------------------------------------------------------------------
% ------------------------------------------------------------------------------
\subsection[Avaliações]{Avaliações}\label{subsec:planejamento-avaliacoes}



% \begin{frame}[allowframebreaks,t]\frametitle{Avaliações ---}
\begin{frame}[t]\frametitle{Avaliações}

  \begin{itemize}
    \justifying{}
    \setlength\itemsep{1em}
    \item Item 1.
    \item Item 2.
    \item Item 3.
  \end{itemize}

\end{frame}



% ------------------------------------------------------------------------------
% ------------------------------------------------------------------------------
% ------------------------------------------------------------------------------
% ------------------------------------------------------------------------------
% ------------------------------------------------------------------------------
\subsection[Comunicação da Turma]{Comunicação da Turma}\label{subsec:planejamento-comunicacao-turma}



% \begin{frame}[allowframebreaks,t]\frametitle{Comunicação da Turma ---}
\begin{frame}[t]\frametitle{Comunicação da Turma}

  \begin{figure}[htb]
    \centering{}
    \includegraphics[width=0.07\paperwidth]{img/sec03/telegram.png}

    \textbf{\href{https://telegram.org}{Telegram}}
  \end{figure}

  \begin{columns}
    \begin{column}{0.45\linewidth}
      \textbf{Vantagens:}
      \begin{itemize}
        \justifying{}
        \setlength\itemsep{1em}
        \item Grupo pode ter muitos membros e a troca de arquivos é bem otimizada.
        \item Histórico permanente, quem entre depois poderá ver todas as mensagens já trocadas no grupo.
        \item Ótimo aplicativo para desktop/web.
      \end{itemize}
    \end{column}

    \begin{column}{0.45\linewidth}
      \textbf{Disponível para:}
      \begin{itemize}
        \justifying{}
        % \setlength\itemsep{1em}
        \item \href{https://telegram.org/dl/android}{Android}
        \item \href{https://telegram.org/dl/ios}{iPhone/iPad}
        \item \href{https://desktop.telegram.org}{PC/Mac/Linux}
        \item \href{https://web.telegram.org}{Web}
        \item \href{https://macos.telegram.org}{macOS}
      \end{itemize}

      \textbf{Link para participar do grupo:}

      \href{https://web.telegram.org}{https://web.telegram.org}
    \end{column}
  \end{columns}

\end{frame}
